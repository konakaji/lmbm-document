\documentclass[12pt]{article}
\usepackage[utf8]{inputenc}
\usepackage{amsmath}
\usepackage{amsfonts}
\usepackage{amssymb}
\usepackage{amsthm}
\usepackage{geometry}
\usepackage{graphicx}
\usepackage{hyperref}
\usepackage{color}
\usepackage{listings}
\usepackage{xcolor}

\geometry{margin=1in}

\title{Transformation from Molecular Orbital Coefficients to Embedding Vectors: Detailed Mathematical Analysis}
\author{EGSMole Project}
\date{\today}

\begin{document}

\maketitle

\section{Overview}

This document provides a detailed mathematical explanation of the transformation process from molecular orbital coefficients (MO coefficients) computed by PySCF to embedding vectors (MO embeddings) that can be used in E3NN (Equivariant Neural Networks).

\section{Overall Transformation Framework}

The transformation from molecular orbital coefficients $\mathbf{C} \in \mathbb{R}^{N_{\text{basis}} \times N_{\text{mo}}}$ to embedding vectors $\mathbf{E} \in \mathbb{R}^{N_{\text{mo}} \times N_{\text{atom}} \times N_{\text{features}}}$ is expressed by the following equation:

\begin{equation}
\mathbf{E} = \text{EmbedMOsOnAtoms}(\mathbf{C}, \mathbf{E}_{\text{energy}}, \mathbf{P}_{\text{atom}}, \mathbf{S}_{\text{irrep}})
\end{equation}

where:
\begin{itemize}
\item $\mathbf{C}$: Molecular orbital coefficient matrix $(N_{\text{basis}} \times N_{\text{mo}})$
\item $\mathbf{E}_{\text{energy}}$: Molecular orbital energies $(N_{\text{mo}})$
\item $\mathbf{P}_{\text{atom}}$: Atom type embedding information
\item $\mathbf{S}_{\text{irrep}}$: Irreducible representation strings for each atom
\end{itemize}

\section{Step-by-Step Transformation Process}

\subsection{Step 1: d-orbital Correction}

Correct the difference in d-orbital ordering between PySCF and E3NN:

\begin{equation}
\mathbf{C}_{\text{fixed}} = \mathbf{F}_D^T \mathbf{C}
\end{equation}

where $\mathbf{F}_D$ is the d-orbital correction matrix.

\subsection{Step 2: Transposition}

Transpose the molecular orbital coefficients to enable independent processing of each molecular orbital:

\begin{equation}
\mathbf{C}_{\text{transposed}} = \mathbf{C}_{\text{fixed}}^T \in \mathbb{R}^{N_{\text{mo}} \times N_{\text{basis}}}
\end{equation}

\subsection{Step 3: Atom-wise Orbital Coefficient Extraction}

For each atom $a$, extract the coefficients corresponding to its basis functions:

\begin{equation}
\mathbf{C}_a = \mathbf{C}_{\text{transposed}}[:, \text{slice}_a]
\end{equation}

where $\text{slice}_a$ is the index range of basis functions corresponding to atom $a$.

\subsection{Step 4: Padding Process}

Pad each atom's orbital coefficients to a unified dimension. Let $\mathbf{S}_a$ be the irreducible representation of atom $a$ and $\mathbf{n}_{\text{target}}$ be the target number of orbitals:

\begin{align}
\mathbf{n}_{\text{count}} &= \text{CountIrreps}(\mathbf{S}_a) \\
\mathbf{n}_{\text{diff}} &= \mathbf{n}_{\text{target}} - \mathbf{n}_{\text{count}} \\
\mathbf{S}_{\text{padded}} &= \mathbf{S}_a + \text{GenerateDiffIrreps}(\mathbf{n}_{\text{diff}})
\end{align}

The padded orbital coefficients are:

\begin{equation}
\mathbf{C}_{a,\text{padded}} = \text{Pad}(\mathbf{C}_a, \mathbf{n}_{\text{diff}})
\end{equation}

\subsection{Step 5: Energy Information Addition}

When molecular orbital energies are provided, add energy information to each atom's embedding vector:

\begin{equation}
\mathbf{E}_{a,\text{with\_energy}} = \text{Concat}(\mathbf{E}_{\text{energy}}, \mathbf{C}_{a,\text{padded}})
\end{equation}

where $\mathbf{E}_{\text{energy}}$ represents the energy values of each molecular orbital.

\subsection{Step 6: Atom Type Information Addition}

Add one-hot encoding of atom type to each atom's embedding vector:

\begin{equation}
\mathbf{A}_{a,\text{one\_hot}} = \text{OneHot}(\text{AtomType}(a), N_{\text{max\_atoms}})
\end{equation}

The final embedding vector is:

\begin{equation}
\mathbf{E}_{a,\text{final}} = \text{Concat}(\mathbf{A}_{a,\text{one\_hot}}, \mathbf{E}_{a,\text{with\_energy}})
\end{equation}

\section{Dimension Calculation}

\subsection{Orbital Contribution Dimension}

The orbital contribution dimension for each atom is determined by the dimension of the irreducible representation after padding:

\begin{equation}
d_{\text{orbital}} = \sum_{l=0}^{l_{\max}} n_l \cdot (2l + 1)
\end{equation}

where:
\begin{itemize}
\item $n_l$: Number of orbitals with angular momentum $l$
\item $2l + 1$: Degeneracy of orbitals with angular momentum $l$
\end{itemize}

\subsection{Energy Dimension}

When molecular orbital energies are provided:

\begin{equation}
d_{\text{energy}} = N_{\text{mo}}
\end{equation}

\subsection{Atom Type Dimension}

One-hot encoding of atom types:

\begin{equation}
d_{\text{atom\_type}} = N_{\text{max\_atoms}}
\end{equation}

\subsection{Total Dimension}

The dimension of the final embedding vector is:

\begin{equation}
N_{\text{features}} = d_{\text{atom\_type}} + d_{\text{orbital}} + d_{\text{energy}}
\end{equation}

\section{Concrete Example: Methane Molecule (CH$_4$)}

\subsection{Input Data}

\begin{itemize}
\item Molecular orbital coefficients: $\mathbf{C} \in \mathbb{R}^{34 \times 34}$
\item Molecular orbital energies: $\mathbf{E}_{\text{energy}} \in \mathbb{R}^{34}$
\item Basis set: def2-svp
\end{itemize}

\subsection{Irreducible Representations for Each Atom}

\begin{align}
\text{Carbon atom}: \quad \mathbf{S}_0 &= \text{1x0e+1x0e+1x0e+1x1o+1x1o+1x2e} \\
\text{Hydrogen atoms}: \quad \mathbf{S}_i &= \text{1x0e+1x0e+1x1o} \quad (i = 1,2,3,4)
\end{align}

\subsection{Padding Process}

Target number of orbitals: $\mathbf{n}_{\text{target}} = [3, 2, 1]$ (s, p, d orbitals)

\begin{align}
\text{Carbon atom}: \quad \mathbf{n}_{\text{count}} &= [3, 2, 1], \quad \mathbf{n}_{\text{diff}} = [0, 0, 0] \\
\text{Hydrogen atoms}: \quad \mathbf{n}_{\text{count}} &= [2, 1, 0], \quad \mathbf{n}_{\text{diff}} = [1, 1, 1]
\end{align}

\subsection{Dimension Calculation}

\begin{align}
d_{\text{orbital}} &= 3 \times 1 + 2 \times 3 + 1 \times 5 = 14 \\
d_{\text{energy}} &= 34 \\
d_{\text{atom\_type}} &= 0 \quad \text{(when energies are included)} \\
N_{\text{features}} &= 0 + 14 + 34 = 48
\end{align}

\subsection{Final Output}

\begin{equation}
\mathbf{E} \in \mathbb{R}^{34 \times 5 \times 48}
\end{equation}

\section{Mathematical Properties}

\subsection{Rotation Invariance}

The embedding vectors possess rotation-invariant properties with respect to molecular rotations:

\begin{equation}
\mathbf{E}(R\mathbf{M}) = \mathbf{E}(\mathbf{M})
\end{equation}

where $R$ is a rotation matrix and $\mathbf{M}$ is the molecular geometry.

\subsection{Permutation Invariance}

Invariant with respect to permutation of identical atoms:

\begin{equation}
\mathbf{E}(\sigma(\mathbf{M})) = \sigma(\mathbf{E}(\mathbf{M}))
\end{equation}

where $\sigma$ represents permutation of identical atoms.

\section{Implementation Considerations}

\subsection{Memory Efficiency}

For large molecules, the embedding vector size becomes significant, making memory-efficient implementation crucial.

\subsection{Numerical Stability}

In the padding process, it is necessary to ensure that zero-padding is numerically stable.

\subsection{Parallel Processing}

Since each atom's processing is independent, parallelization is possible.

\section{Conclusion}

The transformation from molecular orbital coefficients to embedding vectors is a crucial process that converts quantum chemistry calculation results into a format suitable for machine learning. This transformation enables efficient processing of molecular electronic structure information in E3NN.

Each stage of the transformation process is mathematically well-defined, properly converting molecular orbital information into embedding vectors while preserving rotation invariance and permutation invariance.

\end{document}
